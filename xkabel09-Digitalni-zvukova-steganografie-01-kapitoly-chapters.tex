\chapter{Úvod}
\label{cha:introduction}

\todo{obecný popis steganografie}

V~oblasti steganografie je dostupné velké množství kvalitní literatury
popisující teoretické fungování metod, kterými se jednotlivé publikace
zabývají. Je však těžké najít konkrétní implementace popisovaných metod
i~přesto, že autoři je implementovali pro získání výsledků. Proto je hlavním
cílem této práce vytvořit knihovnu v~programovacím jazyce Python, aby měl
kdokoliv možnost pracovat na výzkumu v~této vědecké disciplíně. Někdy totiž
nestačí pouze algoritmický popis různých metod pro pochopení jejich fungování.
Implementace v~konkrétním programovacím jazyce nám přináší formalismus, který
umožňuje snazší pochopení.

V~kapitole~\ref{cha:existing-methods} \uv{Přehled existujících metod digitální
zvukové steganografie} jsou do hloubky popsány steganografické metody vybrané
pro implementaci a~některé další používané. Kapitola~\ref{cha:library-design}
popisuje proč byl vybrán programovací jazyk Python a~použité knihovny. Je zde
také popsána struktura balíčků pro Python a~steganografické metody které byly
vybrány pro implementaci včetně vlastní. Kapitola~\ref{cha:implementation} je
zaměřená na popis rozvržení kódu a~implementaci steganografických metod. Na
konci je popsán způsob vyhodnocení kvality metod, výsledky testování a~srovnání
metod. Poslední kapitola~\ref{cha:conclusion} obsahuje shrnutí výsledků práce.


\chapter{Přehled existujících metod digitální zvukové steganografie}
\label{cha:existing-methods}

\todo{popis kapitoly přehled existujících metod}

\blindtext

\section{Význam steganografie a~její využití}
\label{sec:motivation}

\todo{popis steganografie obecně a jaké má využití}

\blindtext

\blindtext

\blindtext

\section{Reprezentace a~způsob uložení digitálního zvuku}
\label{sec:digital-sound-representation}

\todo{popis způsobu uložení digitálního zvuku}

\blindtext

\blindtext

\section{Vlastnosti metod digitální zvukové steganografie}
\label{sec:method-properties}

\todo{popis vlastností metod}

\blindtext

\blindtext

\begin{figure}[hbt]
    \centering
    \includegraphics[width=0.3\textwidth]{obrazky/placeholder.pdf}
    \caption{Trojúhelník se třemi nejvýznamnějšími vlastnostmi}
    \label{pic:method-property-triangle}
\end{figure}

\section{Metoda nahrazení nejméně významného bitu}
\label{sec:lsb}

\todo{popis metody LSB}

\blindtext

\blindtext

\blindtext

\section{Metoda skrývání pomocí ozvěny}
\label{sec:echo-hiding}

\todo{popis metody Echo hiding}

\blindtext

\blindtext

\subsection*{Ozvěna s~jedním semínkem na bit}
\label{sec:echo-single-kernel}

\todo{popis metody Echo single kernel}

\blindtext

\begin{figure}[hbt]
    \centering
    \includegraphics[width=0.3\textwidth]{obrazky/placeholder.pdf}
    \caption{Ozvěna vzniklá konvolucí s~konvolučními semínky}
    \label{pic:echo-single-kernel-echo}
\end{figure}

\blindtext

\section{Metoda přímého rozprostřeného spektra}
\label{sec:dsss}

\todo{popis metody DSSS}

\blindtext

\begin{figure}[hbt]
    \centering
    \includegraphics[width=0.3\textwidth]{obrazky/placeholder.pdf}
    \caption{Rozložení spektra pomocí metody přímého rozprostření spektra}
    \label{pic:dsss-spreading}
\end{figure}

\blindtext

\section{Metoda fázového kódování}
\label{sec:phase-coding}

\todo{popis metody fázového kódování}

\blindtext

\begin{figure}[hbt]
    \centering
    \includegraphics[width=0.3\textwidth]{obrazky/placeholder.pdf}
    \caption{Změna fáze segmentu pro zakódování bitu}
    \label{pic:phase-coding-phase-change}
\end{figure}

\blindtext

\section{Metoda paritního kódování}
\label{sec:parity-coding}

\todo{popis metody paritního kódování}

\blindtext

\blindtext

\blindtext

\section{Metody založené na nedostatcích lidského sluchového ústrojí}
\label{sec:has}

\todo{popis metody HAS}

\blindtext

\blindtext

\blindtext

\section{Metoda využívající vlnkové transformace}
\label{sec:wavelet-transform}

\todo{popis wavelet metody}

\blindtext


\chapter{Návrh Python knihovny pro zvukovou steganografii}
\label{cha:library-design}

\todo{popis kapitoly návrh knihovny pro steganografii, důvod výběru Pythonu a NumPy}

\blindtext

\section{Struktura balíčků v~programovacím jazyce Python}
\label{sec:python-package-structure}

\todo{popis Python balíčků}

\blindtext

\blindtext

\section{Metody vybrané pro implementaci}
\label{sec:chosen-methods}

\todo{popis vybraných metod}

\blindtext

\blindtext

\section{Popis vlastní metody digitální zvukové steganografie}
\label{sec:own-method}

\todo{popis vlastní metody}

\blindtext

\blindtext


\chapter{Implementace a~testování}
\label{cha:implementation}

\todo{popis kapitoly implementace}

\blindtext

\section{Struktura modulů knihovny}
\label{sec:modules}

\todo{popis modulů knihovny}

\blindtext

\blindtext

\blindtext

\section{Způsoby vyhodnocení kvality metod}
\label{sec:method-quality}

\todo{popis způsobů vyhodnocení kvality}

\blindtext

\blindtext

\blindtext

\section{Výsledky vyhodnocení jednotlivých metod}
\label{sec:method-evaluation}

\todo{výsledky vyhodnocení kvality}

\blindtext

\blindtext

\blindtext


\chapter{Závěr}
\label{cha:conclusion}

\blindtext

\blindtext

\blindtext
