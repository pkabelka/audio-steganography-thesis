\chapter{Úvod}
\label{cha:introduction}

\todo{obecný popis steganografie}

V~oblasti steganografie je dostupné velké množství kvalitní literatury
popisující teoretické fungování metod, kterými se jednotlivé publikace
zabývají. Je však těžké najít konkrétní implementace popisovaných metod
i~přesto, že autoři je implementovali pro získání výsledků. Proto je hlavním
cílem této práce vytvořit knihovnu v~programovacím jazyce Python, aby měl
kdokoliv možnost pracovat na výzkumu v~této vědecké disciplíně. Někdy totiž
nestačí pouze algoritmický popis různých metod pro pochopení jejich fungování.
Implementace v~konkrétním programovacím jazyce nám přináší formalismus, který
umožňuje snazší pochopení.

V~následující kapitole \uv{Přehled existujících metod digitální zvukové
steganografie} jsou do hloubky popsány steganografické metody vybrané pro
implementaci a~některé další používané.

\chapter{Přehled existujících metod digitální zvukové steganografie}
\label{cha:summary}



\chapter{Návrh knihovny pro zvukovou steganografii}
\label{cha:design}



\chapter{Implementace a~testování}
\label{cha:implementation}



\chapter{Závěr}
\label{cha:conclusion}
